\apendice{Anexo de sostenibilización curricular}

\section{Introducción}
Los Objetivos de desarrollo sostenible fueron implantados en 2015 por las naciones unidas. El principal próposito de estos objetivos es que para el año 2030 todas las personas disfruten de paz y prosperidad en este planeta.

Lo que se busca con la persecución de los mismos es lograr un equilibrio social, económico y ambiental en el planeta.

Para la realización de este apartado, se ha realizado una investigación sobre cuales son los diferentes objetivos de desarrollo sostenible y cómo pueden estos verse vinculados con el proyecto realizado.

A continuación se van a mencionar los aspectos en los que se ha considerado que el proyecto contribuye para alcanzar esos objetivos.

\subsection{Objetivo 4 - Educación de calidad}
Este proyecto, facilita el acceso a convocatorias para el Personal Docente e Investigador y Personal de Administración y Servicios. Esto aporta un mayor abanico de oportunidades de empleo en el sector educativo y por lo tanto habrá más opciones que se tengan grupos docentes e investigadores de calidad.

Mediante la utilización de esta web centralizada, habrá un mayor numero de candidatos para una determinada convocatoria y como consecuencia de ello, nos aseguramos de que el profesional elegido, tendrá las condiciones idóneas para realizar esa determinada labor.


\subsection{Objetivo 8 - Trabajo decente y crecimiento económico}

Gracias a web en la que se muestran las convocatorias de manera centralizada, los miembros de la comunidad autónoma incluso del resto del país, tendrán un mejor accesibilidad a determinados empleos lo cual deriva en un crecimiento económico del país.

De esta manera, miembros de otras cuidades que no sean de la ciudad en la que está la propia universdiad, podrán enterarse de puestos y plazas vacantes de los que anteriormente sin esta aplicación no tendrían consciencia de los mismos. Esto permite evitar que haya puestos que se queden sin ocupar y por lo tanto se promueve enormemente el empleo.

Por último, también es importante considerar que los empleos que ofrecen las universidades para PDI y PAS, son empleos con unas condiciones laborables más que óptimas. Esto contribuye a que los empleos sean decentes.

\subsection{Objetivo 9 - Industria, innovación e infraestructura}
Este proyecto es un proyecto de desarrollo software cuyo código es \textit{Open Source}. Esto hace que el código fuente del programa esta disponible para que pueda ser visto, modificado o distribuido por cualquier persona.

Este aspecto contribuye enormemente con este ODS dado que el código abierto permite a otras personas innovar a partir de un proyecto con un código fuente ya desarrollado. De esta manera, una persona que quiera realizar una aplicación innovadora relacionada con esta temática, podrá contar con una base sólida correspondiente a este proyecto.

Por otro lado, la colaboración con la comunidad \textit{Open Source} permite una mayor facilidad para que surgan nuevas empresas y \textit{startups} relacionadas con el desarrollo software.


\subsection{Objetivo 17 - Alianzas para lograr objetivos}
Debido a que la web del proyecto es una web que centraliza convocatorias de distintas universidades, esto potencia enormemente las alianzas entre las distintas universidades de Castilla y León puediendo llegar a compartir recursos y estableciendo convenios entre las mismas en ámbitos docentes, investigadores y administrativos.

