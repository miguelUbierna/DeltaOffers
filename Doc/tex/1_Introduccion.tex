\capitulo{1}{Introducción}

En las universidades españolas, existen dos colectivos que garantizan el desarrollo académico y administrativo. El \textbf{Personal Docente e Investigador}, es el encargado de desempeñar las funciones de docencia, investigación y gestión académica de las universidades. Este colectivo está formado por profesores, investigadores, ayudantes y catedráticos. 

Por otro lado, el \textbf{Personal de Administración y Servicios}, es el encargado de la gestión administrativa, la prestación de servicios y el soporte a la docencia e investigación. Este colectivo esta formado principalmente por administrativos, personal de mantenimiento y servicios y personal técnico.

Ambos colectivos son fundamentales para que el funcionamiento de las universidades sea exitoso. Por esta razón, las universidades necesitan que estos puestos sean desempeñados por personas altamente cualificadas y competentes.

«DeltaOffers», es un proyecto formado por dos aplicaciones independientes. La primera es una aplicación que realiza \textit{web scraping} a las webs de las universidades públicas de Castilla y León para recopilar sus convocatorias y almacenarlas en una base de datos. La segunda, es una aplicación web que tiene como objetivo mostrar todas las convocatorias recopiladas de PDI y PAS de una manera centralizada.

Con la aparición de «DeltaOffers» se espera poner fin a la ausencia de candidatos para plazas de PDI Y PAS de las universidades gracias a que, cualquier persona independientemente de la provincia en la que esté residiendo, podrá enterarse de las convocatorias sin necesidad de tener que acudir a las propias webs de las universidades.

Por otro lado, la existencia de una web centralizada de convocatorias, tiene como consecuencia un mayor alcance de las mismas. Estas convocatorias podrán llegar con una mayor facilidad a los posibles interesados lo que  implicaría un aumento en la demanda de determinados puestos de trabajo y como consecuencia, se espera incrementar la calidad del personal que finalmente sea seleccionado para cubrir una determinada plaza.

Esta web no solo centraliza todas las convocatorias de estas universidades sino que también lo hace con un diseño atractivo y una alta usabilidad proporcionando además funcionalidad adicional como un sistema de avisos por correo electrónico.

Este proyecto podría suponer un gran avance para las universidades públicas de Castilla y León mediante la contribución a un Personal Docente e Investigador y Personal de Administración y Servicios de calidad.

\section{Estructura de la memoria}

La estructura de la memoria es la siguiente:

\begin{itemize}
    \item \textbf{Introducción:}  incluye una descripción del entorno del proyecto y del proyecto en si. Además, se detalla cual es la estructura de la memoria y del resto de materiales entregados.
    \item \textbf{Objetivos del proyecto:} expone cuales son los objetivos que se persiguen con la realización del proyecto. La mayoría de estos objetivos fueron diseñados al comienzo del proyecto y otros se han ido añadiendo en el transcurso del mismo.
    \item \textbf{Conceptos teóricos:} muestra los conceptos teóricos necesarios de comprender para un mejor entendimiento global del proyecto.
    \item \textbf{Técnicas y herramientas:} detalla las técnicas y herramientas utilizadas para el proyecto.
    \item \textbf{Aspectos relevantes del proyecto:}  expone los aspectos más relevantes del proyecto y las soluciones aplicadas en estos casos.  
    \item \textbf{Trabajos relacionados:} presenta algunos proyectos relacionados con el trabajo realizado.
    \item \textbf{Conclusiones y líneas de trabajo futuras:} detalla las conclusiones obtenidas tras la finalización del proyecto junto con un análisis sobre cómo se podría realizar una continuación y mejora del proyecto.    
\end{itemize}

\section{Estructura de los anexos}
\begin{itemize}
    \item \textbf{Plan de proyecto software:} indica cual ha sido la planificación del proyecto junto con un estudio de viabilidad legal y económica del mismo.
    \item \textbf{Especificación de requisitos:} describe cómo se va a comportar el sistema que ha sido desarrollado.
    \item \textbf{Especificación de diseño:} expone la estructura de los datos, los procedimientos utilizados y los patrones de diseño empleados.
    \item \textbf{Documentación técnica de programación:} detalla guías e información técnica del proyecto.
    \item \textbf{Documentación de usuario:} muestra una guía de utilización del producto realizado.  
    \item \textbf{Anexo de sostenibilización curricular:} relaciona el proyecto realizado con los objetivos del desarrollo sostenible.  
\end{itemize}

\section{Estructura del software}
Junto con los anexos y la memoria, se realizará también la entrega del software desarrollado, el cual, tendrá la siguiente estructura.
\begin{itemize}
    \item \textbf{DataCollections:} Este directorio incluye la sección del proyecto en la que se realiza el \textit{web scraping} y el \textit{script} con el que se actualiza la base de datos. Por otro lado, también están aquí ubicados los test unitarios de la parte de recopilación de datos.
    \item \textbf{DeltaOffersWeb:} Esta carpeta contiene el proyecto en .NET mediante el que se desarrolla la página web. En él se encontrarán distintos directorios para controladores, vistas y modelos junto con otros ficheros de ajustes, ejecución , etc.
    \item \textbf{Github Workflows:} Este directorio contiene los archivos de extensión \textit{.yml} los cuales son los encargados de realizar las tareas programadas correspondientes al proyecto mediante \textit{cron}.
\end{itemize}
