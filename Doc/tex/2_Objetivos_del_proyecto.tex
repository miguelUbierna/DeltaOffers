\capitulo{2}{Objetivos del proyecto}
En este capítulo, se van a mostrar los diferentes objetivos marcados para la realización del proyecto.
\section{Objetivos generales}
\begin{itemize}
    \item Recopilar todas las convocatorias de PDI y PAS de las universidades públicas de Castilla y León.
    \item Realizar una web que mostrase de manera centralizada las convocatorias mencionadas.
    \item Mejorar el atractivo, la usabilidad y la estructura frente a las webs que muestran las convocatorias actualmente.
    \item Añadir nuevas funcionalidades en comparación con las webs actuales existentes.
\end{itemize}
\section{Objetivos técnicos}
\begin{itemize}
    \item Recopilación de las convocatorias del PDI Y PAS mediante la realización de \textit{web scraping} con Python a las secciones de las páginas webs en las que las universidades publican las convocatorias.
    \item Tratamiento de los datos obtenidos e implementación de mejoras en cuanto a estructura y formato de los datos para un posterior almacenamiento de manera segura.
    \item Implementación de \textit{testing} mediante el uso de herramientas como \textit{Unittest} o \textit{MagicMock} para la simulación de objetos.
    \item Desarrollo de una página web con ASP .NET Core mediante la utilización de un Patrón MVC (Modelo-Vista-Controlador) que recopilase las convocatorias. 
    \item Añadir funcionalidad a la web mediante un sistema de avisos por correo electrónico y la utilización de ICalendar.
    \item Realizar un despliegue completo de la aplicación mediante servidores Azure y Github Actions.
    \item Utilización de un sistema de control de versiones como Git y utilización de GitHub para alojamiento del código en la nube.
\end{itemize}

\section{Objetivos personales}
\begin{itemize}
    \item Emplear metodologías ágiles como Scrum en un proyecto real.
    \item Utilización de distintas herramientas y aprendizaje del uso de las mismas.
    \item Aplicar los conocimientos adquiridos durante el grado.
    \item Comprobar mi capacidad de resolución de problemas en proyectos de mayor calibre.
    \item Desarrollo y despliegue de mi primera página web.
    \item Mejorar mi habilidad con las herramientas Git y GitHub.
\end{itemize}