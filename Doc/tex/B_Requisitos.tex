\apendice{Especificación de Requisitos}

\section{Introducción}

Los especificación de requisitos en un proyecto software consiste en describir cómo se va a comportar el sistema que ha sido desarrollado. En este apartado, se detallarán los distintos casos de uso los cuales describen las distintas acciones o actividades que va a realizar un actor para que finalmente se lleve a cabo algún proceso.

Tanto los requisitos como los casos de uso son muy importantes en este tipo de proyectos dado que se detallan las funciones que va a realizar el sistema de manera concisa. Además, esto mejora enormemente la comunicación entre el equipo que ha desarrollado el software y el cliente o usuarios finales. Por último, la especificación de estos requisitos cuando se inicia un proyecto, ayuda a localizar los posibles inconvenientes con los que se puede encontrar el equipo de desarrollo a lo largo del mismo.

\section{Objetivos generales}

En este apartado se van a comentar cuales han sido los objetivos del proyecto, algunos de ellos fueron propuestos en las fases iniciales del proyecto, sin embargo, muchos otros se han añadido después para mejorar la funcionalidad de la aplicación web y que esta sea más completa y de mejor calidad. Los objetivos en su totalidad, han sido los siguientes:

\begin{itemize}
\item En las primeras iteraciones, como es lógico, se invirtió gran parte del tiempo a la planificación y estructuración del proyecto.

\item Recopilación de convocatorias de Personal Docente Investigador y Personal de Administración y Servicios de las Universidades publicas de Castilla y León de sus respectivas páginas web. 

\item Interpretar los datos obtenidos, tratamiento de los mismos y almacenarlos de manera segura.

\item Proporcionar a los usuarios interesados en este tipo de ofertas, una web unificada que listase todas estas convocatorias con un formato unificado. Además, esta web mejorará en usabilidad y en funcionalidad a las webs en las que las universidades publican sus convocatorias.

\item Proporcionar el acceso a información mas detallada de estas convocatorias junto con un enlace que te redirecciona a la web original.

\item Permitir que los usuarios puedan recibir correos con un archivo adjunto para que puedan añadir las convocatorias a sus respectivos calendarios.

\end{itemize}

\section{Catálogo de requisitos}

Una vez mencionados los objetivos, se enumerarán los requisitos funcionales y no funcionales del proyecto:

\textbf{Requisitos funcionales:}

\begin{itemize}
\item\textbf{RF-1 Recopilación de datos}: La aplicación tiene que ser capaz de recopilar datos de las distintas webs mediante \textit{web scraping}.
\begin{itemize}
\item\textbf{RF-1.1 Tratamiento de datos}: La aplicación debe ser capaz de tratar los datos e interpretarlos.
\begin{itemize}
\item\textbf{RF-1.1.1 Estructuración de datos}: La aplicación debe estructurar los datos obtenidos para que estos tengan un formato y estructura uniforme.
\end{itemize}
\item\textbf{RF-1.2 Almacenamiento de datos}: La aplicación debe almacenar los datos recopilados de forma segura en una base de datos.
\item\textbf{RF-1.3 Actualización de los datos}: Los datos mostrados en la aplicación deben de ser actualizados periodicamente con las últimas actualizaciones en las webs de origen.
\end{itemize}
\end{itemize}

\begin{itemize}
\item\textbf{RF-2 Visualización de las convocatorias}: La aplicación tiene que ser capaz de mostrar la información de las distintas convocatorias.
\begin{itemize}
\item\textbf{RF-2.1 Mostrar listado maestro de las convocatorias}: La aplicación debe de listar en una pagina \textit{maestra} la información esencial y descriptiva de cada convocatoria, esto se realizará de manera paginada.
\item\textbf{RF-2.2 Aplicación de filtros}: La web tiene que permitir la aplicación de filtros para que los usuarios puedan acceder más rápido a las convocatorias de interés.
\item\textbf{RF-2.3 Mostrar Detalles}: La aplicación debe de tener páginas \textit{detalle} con los detalles de las convocatorias que el usuario selecciona.

\end{itemize}

\item\textbf{RF-3 Acceso al portal de convocatorias de las universidades}: La aplicación tiene que permitir al usuario acceder mediante un enlace en la página de detalles a las paginas webs propias de cada universidad.

\item\textbf{RF-4 Activar avisos}: El usuario debe poder activar avisos en las convocatorias que le interesen.

\begin{itemize}
\item\textbf{RF-4.1 Suscripción a avisos por correo}: El usuario debe poder suscribirse a recibir avisos por medio del correo electrónico.
\begin{itemize}
\item\textbf{RF-4.1.1 Comprobación correo válido }: La aplicación debe comprobar si el correo con el que el usuario se ha suscrito es valido.
\end{itemize}
\item\textbf{RF-4.2 Recibir avisos}: La aplicación debe enviar avisos por correo de las convocatorias que el usuario desea y el usuario deberá recibir estos.
\item\textbf{RF-4.3 Recibir archivos ICalendar adjuntos}: La aplicación debe generar un archivo ICalendar con la fecha fin de las convocatorias y adjuntarlo en el correo enviado al usuario.
\begin{itemize}
\item\textbf{RF-4.3.1 Compatibilidad ICalendar con principales servicios de calendario}: La aplicación debe proporcionar un de extension ".ics" que debe de ser compatible con Outlook, Google Calendar y otros.
\end{itemize}
\item\textbf{RF-4.4 Acceso al portal de convocatorias de las universidades}: En el correo electronico que recibe el usuario se debe permitir acceder mediante un enlace en la página de detalles a las paginas webs propias de cada universidad.
\end{itemize}
\item\textbf{RF-5 Comienzo de uso de la web}: La aplicación debe mostrar una ventana principal con información de la tematica de la web junto con un acceso para comenzar su uso.
\end{itemize}

\textbf{Requisitos no funcionales:}

\begin{itemize}
\item\textbf{RNF-1 Rendimiento:} Los tiempos de respuesta son aceptables contando que se está utilizando un servidor web gratuito. La aplicación siempre responde ante las peticiones correctamente aunque el tiempo de respuesta en ocasiones no sea inmediato.
\item\textbf{RNF-2 Seguridad:} Los
datos obtenidos serán almacenados de manera segura sin que se pueda acceder de manera no autorizada. 
\item\textbf{RNF-3 Usabilidad:} La experiencia del usuario al interactuar con el sistema es buena. La web es intuitiva y facil de utilizar. La curva de aprendizaje es muy pequeña. 
\item\textbf{RNF-4 Disponibilidad:} La aplicación web esta disponible para los usuarios siempre y cuando tengan conexión a internet.
\item\textbf{RNF-5 Mantenibilidad:} La aplicación web será facil de mantener siempre y cuando la estructura HTML de las webs de las que se extraen los datos sea la misma. En el caso de que no, la web necesitará de soporte. La implementación de test unitarios facilita enormement este aspecto.
\item\textbf{RNF-6 Escalabilidad:} La estructura del proyecto esta diseñada para que la aplicación sea facilmente escalable gracias a la aplicación de buenas prácticas y patrones de diseño.
\item\textbf{RNF-7 Portabilidad:} La pagina web puede ser accedida desde cualquier navegador. Además, tiene un diseño adaptativo para que sea compatible con multitud de dispositivos independientemente del ancho de la pantalla.

\end{itemize}





\section{Especificación de requisitos}
En este apartado se mostrará un diagrama de los casos de uso basandonos en los requisitos previamente mencionados. Posteriormente, se detallarán cada uno de estos.

\newpage
\begin{landscape}
\subsection{Diagrama de casos de uso}\label{diagrama-de-casos-de-uso}
\vspace{1cm}
\imagenapaisada{DiagramaCasosUsoTFG}{Diagrama de casos de uso.}
\end{landscape}
\newpage

\subsection{Actores}
Los actores en un diagrama de casos de uso, son aquellos que desempeñan alguna función dentro del sistema. Los actores no tienen porque ser una sola persona sino también un equipo o un sistema externo. En el caso de nuestro sistema, hemos encontrado gran diversidad en cuanto a los actores:

\begin{itemize}
\item\textbf{Usuario}: Son aquellos que van a utilizar la aplicación y dispondrán de toda la funcionalidad de esta. La aplicación esta diseñada para que el usuario pueda interactuar con ella de una manera sencilla y con una buena experiencia de usuario.
\item\textbf{Equipo desarrollo/soporte}: En este proyecto, los desarrolladores y el equipo de soporte, también realizan acciones que impactan en el sistema. Esto se debe a que, dado que nuestra web esta constantemente recopilando datos de webs externas, en el caso de que nuevas convocatorias de nuevas universidades se quieran mostrar en la web o que cambie la estructura de la actual web de origen, el equipo de desarrollo y soporte tendrán que realizar labores de tratamiento de datos, estructuración y almacenamiento.
\item\textbf{Sistema cron}: Este sistema ha sido el encargado de actualizar la base de datos periodicamente según se iban actualizando las convocatorias en las webs de las propias universidades.De esta manera, la información en la web del proyecto, estaba actualizada constantemente.
\item\textbf{Sistema correo}: Por último, otro de los actores involucrados ha sido el sistema de correo. Este ha permitido el envio de avisos a los usuarios que van a utilizar la web. En este caso, el sistema utilizado ha sido el servidor de correo electrónico SMTP, el cual permite el envío de emails de manera automatizada.
\end{itemize}


\begin{table}[p]
	\centering
	\begin{tabularx}{\linewidth}{ p{0.21\columnwidth} p{0.71\columnwidth} }
		\toprule
		\textbf{CU-1}    & \textbf{Recopilación de datos}\\
		\toprule
		\textbf{Versión}              & 1.0    \\
		\textbf{Autor}                & Miguel Ubierna Gutiérrez \\
		\textbf{Requisitos asociados} & RF-1, RF-1.1, RF-1.1.1, RF-1.2, RF-1.3,  \\
		\textbf{Descripción}          & La aplicación deberá disponer de las convocatorias que las universidades muestran en sus propias webs. \\
		\textbf{Precondición}         & Las webs de las correspondientes universidades deberán estar disponibles. \\
		\textbf{Acciones}             &
		\begin{enumerate}
			\def\labelenumi{\arabic{enumi}.}
			\tightlist
			\item El equipo de desarrollo examinará la estructura de las webs para su extracción de datos.
			\item Recopilará los datos más interesantes
                \item Se tratarán los datos recopilados y se almacenarán los mismos.
                \item Los datos se mostrarán de manera actualizada en la web mediante el sistema cron.
		\end{enumerate}\\
		\textbf{Postcondición}        & La base de datos será actualizada. \\
		\textbf{Excepciones}          & Fallo en las solicitudes HTTP a las correspondientes webs. \\
		\textbf{Importancia}          & Alta. \\
		\bottomrule
	\end{tabularx}
	\caption{CU-1 Recopilación de datos.}
\end{table}


\begin{table}[p]
	\centering
	\begin{tabularx}{\linewidth}{ p{0.21\columnwidth} p{0.71\columnwidth} }
		\toprule
		\textbf{CU-2}    & \textbf{Tratamiento de datos}\\
		\toprule
		\textbf{Versión}              & 1.0    \\
		\textbf{Autor}                & Miguel Ubierna Gutiérrez \\
		\textbf{Requisitos asociados} & RF-1.1, RF-1.1.1 \\
		\textbf{Descripción}          & La aplicación debe ser capaz de tratar los datos, para que solo sean mostrados aquellos que sean descriptivos y de interés. \\
		\textbf{Precondición}         & Que se hayan podido recopilar los datos correctamente. \\
		\textbf{Acciones}             &
		\begin{enumerate}
			\def\labelenumi{\arabic{enumi}.}
			\tightlist
			\item Se obtienen todos los datos.
			\item Se detectan los datos no válidos o erroneos
                \item Se detectan que no son de interés.
                \item Se trabaja con los datos de interés
		\end{enumerate}\\
		\textbf{Postcondición}        & Ninguna. \\
		\textbf{Excepciones}          & Ninguna. \\
		\textbf{Importancia}          & Alta.  \\
		\bottomrule
	\end{tabularx}
	\caption{CU-2 Tratamiento de datos.}
\end{table}


\begin{table}[p]
	\centering
	\begin{tabularx}{\linewidth}{ p{0.21\columnwidth} p{0.71\columnwidth} }
		\toprule
		\textbf{CU-3}    & \textbf{Estructuración de datos}\\
		\toprule
		\textbf{Versión}              & 1.0    \\
		\textbf{Autor}                & Miguel Ubierna Gutiérrez \\
		\textbf{Requisitos asociados} & RF-1.1.1 \\
		\textbf{Descripción}          & La aplicación debe ser capaz de tratar los datos, para que después estos sean mostrados con una estructura y formato correctos. \\
		\textbf{Precondición}         & Se debe disponer de los datos de interés. \\
		\textbf{Acciones}             &
		\begin{enumerate}
			\def\labelenumi{\arabic{enumi}.}
			\tightlist
			\item Se obtienen los datos de interés.
			\item Se estructuran los mismos por categorias o importancia.
                \item Se corrigen las divergencias en cuanto a los formatos
                \item Se almacenan los mismos de manera correcta.
		\end{enumerate}\\
		\textbf{Postcondición}        & La base de datos debe estar disponible. \\
		\textbf{Excepciones}          & Ninguna.  \\
		\textbf{Importancia}          & Media.  \\
		\bottomrule
	\end{tabularx}
	\caption{CU-3 Estructuración de datos.}
\end{table}

\begin{table}[p]
	\centering
	\begin{tabularx}{\linewidth}{ p{0.21\columnwidth} p{0.71\columnwidth} }
		\toprule
		\textbf{CU-4}    & \textbf{Almacenamiento de datos}\\
		\toprule
		\textbf{Versión}              & 1.0    \\
		\textbf{Autor}                & Miguel Ubierna Gutiérrez \\
		\textbf{Requisitos asociados} & RF-1.2 \\
		\textbf{Descripción}          & La aplicación debe ser capaz de almacenar los datos de interés que ya tienen el formato adecuado. \\
		\textbf{Precondición}         & Disponer de una base de datos.\\
		\textbf{Acciones}             &
		\begin{enumerate}
			\def\labelenumi{\arabic{enumi}.}
			\tightlist
			\item Se configura un servidor para almacenar la base de datos
			\item Se crean las bases de datos en el servidor.
                \item La aplicación se conecta con el servidor para transferir sus datos al mismo.
                \item Se almacenan los datos de manera correcta.
		\end{enumerate}\\
		\textbf{Postcondición}        & El servidor debe estar disponible. \\
		\textbf{Excepciones}          & El servidor no esta disponible, no se cargarán las convocatorias.  \\
		\textbf{Importancia}          & Alta.  \\
		\bottomrule
	\end{tabularx}
	\caption{CU-4 Almacenamiento de datos.}
\end{table}

\begin{table}[p]
	\centering
	\begin{tabularx}{\linewidth}{ p{0.21\columnwidth} p{0.71\columnwidth} }
		\toprule
		\textbf{CU-5}    & \textbf{Actualización de los datos}\\
		\toprule
		\textbf{Versión}              & 1.0    \\
		\textbf{Autor}                & Miguel Ubierna Gutiérrez \\
		\textbf{Requisitos asociados} & RF-1.3 \\
		\textbf{Descripción}          & La aplicación debe ser capaz de mostrar las ofertas de manera actualizada basandose en las webs sobre las que se extraen los datos. \\
		\textbf{Precondición}         & Disponer conexión con el servidor de base de datos.\\
		\textbf{Acciones}             &
		\begin{enumerate}
			\def\labelenumi{\arabic{enumi}.}
			\tightlist
			\item Se configura un sistema cron para que los datos se actualicen de manera periodica.
			\item El sistema cron ejecuta determinados scripts de la aplicación para que la web este actualizada.
                \item Se almacenan los datos actualizados en la base de datos.
		\end{enumerate}\\
		\textbf{Postcondición}        & Las ofertas estarán actualizadas. \\
		\textbf{Excepciones}          & El servidor no esta disponible, no se cargarán las nuevas convocatorias.  \\
		\textbf{Importancia}          & Alta.  \\
		\bottomrule
	\end{tabularx}
	\caption{CU-5 Actualización de los datos.}
\end{table}


\begin{table}[p]
	\centering
	\begin{tabularx}{\linewidth}{ p{0.21\columnwidth} p{0.71\columnwidth} }
		\toprule
		\textbf{CU-6}    & \textbf{Visualización de las convocatorias}\\
		\toprule
		\textbf{Versión}              & 1.0    \\
		\textbf{Autor}                & Miguel Ubierna Gutiérrez \\
		\textbf{Requisitos asociados} & RF-2, RF-2.1, RF-2.2, RF-2.3, \\
		\textbf{Descripción}          & Permite que el usuario pueda visualizar las convocatorias recopiladas de las distintas webs de las universidades.  \\
		\textbf{Precondición}         & Disponer conexión con el servidor de base de datos.\\
		\textbf{Acciones}             &
		\begin{enumerate}
			\def\labelenumi{\arabic{enumi}.}
			\tightlist
   			\item Se cargan las respectivas convocatorias de las bases de datos.
			\item El usuario empieza con el uso de la aplicación.
			\item El usuario podrá ver las convocatorias.
		\end{enumerate}\\
		\textbf{Postcondición}        & Ninguna. \\
		\textbf{Excepciones}          & El servidor que aloja la base de datos no está disponible, no se cargarán las convocatorias.  \\
		\textbf{Importancia}          & Alta.  \\
		\bottomrule
	\end{tabularx}
	\caption{CU-6 Visualización de las convocatorias.}
\end{table}


\begin{table}[p]
	\centering
	\begin{tabularx}{\linewidth}{ p{0.21\columnwidth} p{0.71\columnwidth} }
		\toprule
		\textbf{CU-7}    & \textbf{Mostrar listado maestro de las convocatorias}\\
		\toprule
		\textbf{Versión}              & 1.0    \\
		\textbf{Autor}                & Miguel Ubierna Gutiérrez \\
		\textbf{Requisitos asociados} & RF-2.1 \\
		\textbf{Descripción}          & Permite al usuario visualizar un listado maestro de las convocatorias con los atributos más descriptivos de las mismas.  \\
		\textbf{Precondición}         & Disponer  de conexión con el servidor de base de datos.\\
		\textbf{Acciones}             &
		\begin{enumerate}
			\def\labelenumi{\arabic{enumi}.}
			\tightlist
			\item El usuario accede a la página maestra de la aplicación.
                \item Se cargan las respectivas convocatorias de las bases de datos.
			\item Se mostrará un listado paginado de las convocatorias con los atributos mas descriptivos de las mismas como la categoría, plazo y universidad.
		\end{enumerate}\\
		\textbf{Postcondición}        & Ninguna. \\
		\textbf{Excepciones}          & El servidor que aloja la base de datos no esta disponible, no se cargarán las convocatorias.  \\
		\textbf{Importancia}          & Alta.  \\
		\bottomrule
	\end{tabularx}
	\caption{CU-7 Mostrar listado maestro de las convocatorias.}
\end{table}


\begin{table}[p]
	\centering
	\begin{tabularx}{\linewidth}{ p{0.21\columnwidth} p{0.71\columnwidth} }
		\toprule
		\textbf{CU-8}    & \textbf{Aplicación de filtros}\\
		\toprule
		\textbf{Versión}              & 1.0    \\
		\textbf{Autor}                & Miguel Ubierna Gutiérrez \\
		\textbf{Requisitos asociados} & RF-2.2 \\
		\textbf{Descripción}          & Permite al usuario aplicar los filtros que este desea para que encuentre de manera rápida las convocatorias de su interés.  \\
		\textbf{Precondición}         & Las convocatorias deben de cargar correctamente.\\
		\textbf{Acciones}             &
		\begin{enumerate}
			\def\labelenumi{\arabic{enumi}.}
			\tightlist
			\item Se mostrará un listado paginado de las convocatorias con los atributos mas descriptivos de las mismas como la categoría, plazo y universidad.
                \item El usuario podrá filtrar por estos campos.
                \item Una vez se pulse al botón 'Filtrar' se aplicarán los filtros.
                \item Se mostrarán las convocatorias filtradas y paginadas.
		\end{enumerate}\\
		\textbf{Postcondición}        & Las convocatorias devueltas deben de ser solo las que cumplen el filtro. \\
		\textbf{Excepciones}          & Ninguna.  \\
		\textbf{Importancia}          & Media.  \\
		\bottomrule
	\end{tabularx}
	\caption{CU-8 Aplicación de filtros.}
\end{table}


\begin{table}[p]
	\centering
	\begin{tabularx}{\linewidth}{ p{0.21\columnwidth} p{0.71\columnwidth} }
		\toprule
		\textbf{CU-9}    & \textbf{Mostrar detalles}\\
		\toprule
		\textbf{Versión}              & 1.0    \\
		\textbf{Autor}                & Miguel Ubierna Gutiérrez \\
		\textbf{Requisitos asociados} & RF-2.3 \\
		\textbf{Descripción}          & Permite al usuario ver detalles de las convocatorias por las que está interesado con información adicional a la que se mostraba en el listado maestro.  \\
		\textbf{Precondición}         & Las convocatorias deben de cargar correctamente.\\
		\textbf{Acciones}             &
		\begin{enumerate}
			\def\labelenumi{\arabic{enumi}.}
			\tightlist
			\item El usuario detecta en el listado una convocatoria que es de su interés.
                \item Accede a los detalles de la misma.
                \item Se mostrará información más detallada de la convocatoria junto con los atributos mostrados en el listado.
		\end{enumerate}\\
		\textbf{Postcondición}        & Se deberán mostrar los detalles de la convocatoria que ha seleccionado el usuario. \\
		\textbf{Excepciones}          & Ninguna  \\
		\textbf{Importancia}          & Media  \\
		\bottomrule
	\end{tabularx}
	\caption{CU-9 Mostrar detalles.}
\end{table}

\begin{table}[p]
	\centering
	\begin{tabularx}{\linewidth}{ p{0.21\columnwidth} p{0.71\columnwidth} }
		\toprule
		\textbf{CU-10}    & \textbf{Acceso al portal de convocatorias de las universidades}\\
		\toprule
		\textbf{Versión}              & 1.0    \\
		\textbf{Autor}                & Miguel Ubierna Gutiérrez \\
		\textbf{Requisitos asociados} & RF-3, RF-4.4 \\
		\textbf{Descripción}          & Permite a los usuarios acceder mediante un enlace a las secciones de las webs donde las propias universidades suben sus convocatorias.  \\
		\textbf{Precondición}         & Los detalles de las convocatorias deben haber cargado correctamente.\\
		\textbf{Acciones}             &
		\begin{enumerate}
			\def\labelenumi{\arabic{enumi}.}
			\tightlist
			\item El usuario accede a los detalles de una convocatoria desde el listado general.
                \item Dentro de la ventana detalles habrá un enlace que redigirá al usuario a la propia web de la universidad correspondiente.
                \item El usuario visualizará la convocatoria correspondiente y podrá ofrecerse como candidato realizando los trámites requeridos por la universidad.
                \item El usuario también podrá acceder desde el correo cuando realiza una suscripción para recibir avisos.                
		\end{enumerate}\\
		\textbf{Postcondición}        & Niguna \\
		\textbf{Excepciones}          & Ninguna  \\
		\textbf{Importancia}          & Alta  \\
		\bottomrule
	\end{tabularx}
	\caption{CU-10 Acceso al portal de convocatorias de las universidades.}
\end{table}


\begin{table}[p]
	\centering
	\begin{tabularx}{\linewidth}{ p{0.21\columnwidth} p{0.71\columnwidth} }
		\toprule
		\textbf{CU-11}    & \textbf{Activar Avisos}\\
		\toprule
		\textbf{Versión}              & 1.0    \\
		\textbf{Autor}                & Miguel Ubierna Gutiérrez \\
		\textbf{Requisitos asociados} & RF-4, RF-4.1, RF-4.1.1, RF-4.2, RF-4.3, RF-4.3.1, RF-4.4  \\
		\textbf{Descripción}          & Permite que el sistema de correo pueda enviar avisos sobre las convocatorias a los usuarios.  \\
		\textbf{Precondición}         & El listado de las convocatorias ha debido de cargar correctamente.\\
		\textbf{Acciones}             &
		\begin{enumerate}
			\def\labelenumi{\arabic{enumi}.}
			\tightlist
			\item El usuario visualiza el listado maestro de las convocatorias.
                \item Selecciona la opción de obtener avisos en la convocatoria en la que está interesado.
                \item Si la validación es correcta, se enviará el aviso correspondiente.
		\end{enumerate}\\
		\textbf{Postcondición}        & El email debe ser enviado al correo electrónico proporcionado. \\
		\textbf{Excepciones}          & Ventana de error en el caso de que el correo no sea correcto.  \\
		\textbf{Importancia}          & Baja.  \\
		\bottomrule
	\end{tabularx}
	\caption{CU-11 Activar Avisos.}
\end{table}


\begin{table}[p]
	\centering
	\begin{tabularx}{\linewidth}{ p{0.21\columnwidth} p{0.71\columnwidth} }
		\toprule
		\textbf{CU-12}    & \textbf{Suscripción a avisos por correo}\\
		\toprule
		\textbf{Versión}              & 1.0    \\
		\textbf{Autor}                & Miguel Ubierna Gutiérrez \\
		\textbf{Requisitos asociados} & RF-4.1, RF-4.1.1  \\
		\textbf{Descripción}          & Permite que el usuario se suscriba al sistema de avisos de la app para recibir avisos de las convocatorias que el desee.  \\
		\textbf{Precondición}         & Ninguna.\\
		\textbf{Acciones}             &
		\begin{enumerate}
			\def\labelenumi{\arabic{enumi}.}
			\tightlist
			\item El usuario indica que quiere recibir avisos de una determinada convocatoria.
                \item El usuario deberá rellenar un formulario con su email y aceptar recibir emails.
                \item Una vez realizado esto, el usuario ya estará suscrito para recibir avisos.
		\end{enumerate}\\
		\textbf{Postcondición}        & Ninguna. \\
		\textbf{Excepciones}          & Ventana error en el caso de suscripción incorrecta.  \\
		\textbf{Importancia}          & Baja.  \\
		\bottomrule
	\end{tabularx}
	\caption{CU-12 Suscripción a avisos por correo.}
\end{table}


\begin{table}[p]
	\centering
	\begin{tabularx}{\linewidth}{ p{0.21\columnwidth} p{0.71\columnwidth} }
		\toprule
		\textbf{CU-13}    & \textbf{Comprobación correo valido}\\
		\toprule
		\textbf{Versión}              & 1.0    \\
		\textbf{Autor}                & Miguel Ubierna Gutiérrez \\
		\textbf{Requisitos asociados} & RF-4.1.1  \\
		\textbf{Descripción}          & La aplicación dispone de un sistema de validación de correos para asegurarse de que el usuario completa correctamente el formulario.  \\
		\textbf{Precondición}         & Ninguna.\\
		\textbf{Acciones}             &
		\begin{enumerate}
			\def\labelenumi{\arabic{enumi}.}
			\tightlist
			\item El usuario acepta la recepción de emails para ser avisado de una determinada convocatoria.
                \item Si el email es valido, se enviará el aviso.
                \item Se mostrará una alerta en la que indica que el aviso ha sido enviado correctamente.
		\end{enumerate}\\
		\textbf{Postcondición}        & Ninguna. \\
		\textbf{Excepciones}          & Ventana error en el caso de que el correo no sea válido. \\
		\textbf{Importancia}          & Baja.  \\
		\bottomrule
	\end{tabularx}
	\caption{CU-13 Comprobación correo valido.}
\end{table}


\begin{table}[p]
	\centering
	\begin{tabularx}{\linewidth}{ p{0.21\columnwidth} p{0.71\columnwidth} }
		\toprule
		\textbf{CU-14}    & \textbf{Recibir archivos ICalendar adjuntos}\\
		\toprule
		\textbf{Versión}              & 1.0    \\
		\textbf{Autor}                & Miguel Ubierna Gutiérrez \\
		\textbf{Requisitos asociados} & RF-4.3  \\
		\textbf{Descripción}          & El sistema de correo enviará archvos ICalendar adjuntos con la fecha de fin de las convocatorias para que los usuarios puedan añadirlo a sus calendarios personales.  \\
		\textbf{Precondición}         & Ninguna.\\
		\textbf{Acciones}             &
		\begin{enumerate}
			\def\labelenumi{\arabic{enumi}.}
			\tightlist
			\item El usuario se suscribe a los avisos correctamente.
                \item El usuario recibe un email con el aviso y el archivo ICalendar.
                \item El archivo ICalendar puede ser abierto por el usuario y añadirlo a su calendario personal.
		\end{enumerate}\\
		\textbf{Postcondición}        & Deberá aparecer un nuevo evento en el calendario personal del usuario. \\
		\textbf{Excepciones}          & Ninguna. \\
		\textbf{Importancia}          & Baja.  \\
		\bottomrule
	\end{tabularx}
	\caption{CU-14 Recibir archivos ICalendar adjuntos.}
\end{table}


\begin{table}[p]
	\centering
	\begin{tabularx}{\linewidth}{ p{0.21\columnwidth} p{0.71\columnwidth} }
		\toprule
		\textbf{CU-15}    & \textbf{Compatibilidad ICalendar}\\
		\toprule
		\textbf{Versión}              & 1.0    \\
		\textbf{Autor}                & Miguel Ubierna Gutiérrez \\
		\textbf{Requisitos asociados} & RF-4.3.1  \\
		\textbf{Descripción}          & El archivo de extension .ics deberá ser compatible con los principales servicios de calendario.\\
		\textbf{Precondición}         & El archivo ICalendar se ha debido de generar correctamente.\\
		\textbf{Acciones}             &
		\begin{enumerate}
			\def\labelenumi{\arabic{enumi}.}
			\tightlist
                \item El usuario abre el archivo con su correo.
                \item El evento se añadira a los servicios de calendario más populares como Google Calendar o Outlook.
		\end{enumerate}\\
		\textbf{Postcondición}        & Deberá aparecer un nuevo evento en el calendario personal del usuario. \\
		\textbf{Excepciones}          & Si el formato no es compatible, le aparecerá una excepción al usuario. \\
		\textbf{Importancia}          & Baja.  \\
		\bottomrule
	\end{tabularx}
	\caption{CU-15 Compatibilidad ICalendar.}
\end{table}


