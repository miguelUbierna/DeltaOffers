\apendice{Plan de Proyecto Software}

\section{Introducción}
En los años actuales, se viene dando gran prioridad a dedicar tiempo y recursos para realizar una  planificación del proyecto de calidad. Esto se debe a que en todo proyecto es muy importante asentar unas bases sólidas de planificación y de estructura antes de ponernos a trabajar en los aspectos más técnicos.

Este plan de proyecto nos indicará los pasos a seguir para que el resultado de un proyecto sea satisfactorio y por lo tanto se consiga alcanzar el éxito.

Esta fase de planificación va a estar fragmentada en dos partes:
\begin{itemize}
\item 
\textbf{Planificación temporal: } Donde se detallarán los plazos que se han seguido en el proyecto y las distintas tareas y actividades realizadas en estos mismos.
\item 
\textbf{Estudio de viabilidad: } Donde se realizará un estudio económico con su respectivo análisis de costes y beneficios y, por otro lado, un estudio legal en el que se analizarán todas las leyes que pueden llegar a estar involucradas en el proyecto.
\end{itemize}

Ambas partes deben de tenerse en cuenta a la hora de realizar un proyecto y si se realizan correctamente, estaremos en una situación ventajosa para llevar a cabo el proyecto con el menor número de contratiempos y problemas posibles. 
\section{Planificación temporal}
La gestión de este proyecto se ha llevado a cabo siguiendo una metodología Scrum~\cite{scrum:latex}, metodología basada en \textit{sprints}. Durante estos \textit{sprints} de dos semanas de duración, se acordaba una reunión retrospectiva con los tutores en la cual se hacían revisiones de las tareas realizadas durante el \textit{sprint} y se planteaban nuevas tareas para \textit{sprints} futuros.

Todas estas tareas abordadas, han sido recogidas dentro del apartado \textit{Issues} en la plataforma de desarrollo colaborativo con control de versiones basado en Git (Github). Además, durante estos \textit{spints}, también se iban subiendo a esta misma plataforma, los distintos desarrollos realizados.

Por otro lado, cabe destacar que la metodología ágil Scrum no se ha podido realizar en su totalidad dado que es una metodología diseñada para grupos de trabajo y este proyecto ha sido un proyecto individual.

Finalmente, se van a detallar cuáles han sido los \textit{sprints} realizados con algunas descripciones de las tareas y actividades realizadas. Además, se mostrarán una serie de gráficas con el porcentaje de tiempo dedicado en el sprint a las distintas actividades realizadas.

\subsection{Sprint 1 (08/02/2024 -
21/02/2024)}

Este \textit{sprint} comenzó con una reunión inicial con mis tutores César Ignacio García Osorio y Ana Serrano Mamolar tras haber acordado su mentorización para este proyecto previamente. En esta reunión se determinaron los objetivos del proyecto realizando una estructuración de este.

Por otro lado, también se trató el tema de las herramientas que iban a ser utilizadas para la realización del proyecto, buscando unas herramientas que se adaptasen a las tareas que se iban a realizar y al objetivo final que se quería conseguir. 

Finalmente, se me dieron algunos consejos en función de las dudas que tenía sobre como estructurar y organizar el proyecto en cuanto a tiempos y a las actividades a realizar.


El trabajo realizado estas dos semanas consistió en:
\begin{itemize}
\item 
\textbf{Investigación de las webs de las que voy a extraer información mediante \textit{web scraping}: } Para determinar todos los atributos candidatos a recopilar para mostrarlos en la web de mi proyecto.
\item 
\textbf{Determinar las bibliotecas de Python que van a ser utilizadas para llevar a cabo el proceso de \textit{web scraping} }

\end{itemize}

\imagensprint{sprint1}{Sprint 1.}

\subsection{Sprint 2 (22/02/2024 -
06/03/2024)}
Durante esta segunda iteración se realizó el proceso de \textit{web scraping} para las universidades públicas de Castilla y León. Esta fue una primera toma de contacto con el \textit{scraping} en la que se obtuvieron los atributos de cada una de las convocatorias y para cada una de las universidades.

Para este proceso se utilizaron herramientas como Requests ~\cite{requests:latex}, Selenium ~\cite{selenium:latex} y BeautifulSoup ~\cite{beautifulsoup:latex}. Durante este \textit{sprint} se fue aprendiendo la utilización de estas herramientas mediante la lectura de documentación y su aplicación práctica.

Además, dado que la calidad de algunos datos cuando se realizó el \textit{scraping} no eran buenos, se decidió realizar la limpieza de algunos de ellos ya pensando en la calidad de mi web en el futuro.

Por último, se realizó una subida al repositorio del proyecto en GitHub con los desarrollos realizados durante este \textit{sprint}.

\imagensprint{sprint2}{Sprint 2.}

\subsection{Sprint 3 (07/03/2024 -
20/03/2024)}
Por un lado, este periodo consistió en añadir funcionalidad nueva al proceso de \textit{web scraping} mediante la explotación de nuevos atributos los cuales no habían sido contemplados anteriormente.

Sin embargo, también se buscó mucho mejorar la calidad del código realizado hasta esta fecha. Para ello se decidió pasar el desarrollo realizado a programación orientada a objetos teniendo así una estructura mucho más definida y favoreciendo de esta manera la eliminación de duplicidad de código. De esta manera, se estructuró el proyecto con una clase por cada una de las universidades de las que se extrajeron los datos.

Por último, se aplicó la guía de estilos de \textit{PEP8}  para mejorar la legibilidad y calidad del código. Esto se hizo con una herramienta disponible para Python llamada Autopep8~\cite{autopep8:latex} la cual nos aplicaba la guía de estilos al guardar los cambios en el desarrollo realizado.

\imagensprint{sprint3}{Sprint 3.}

\subsection{Sprint 4 (21/03/2024 -
03/04/2024)}
En esta iteración, se realizó una limpieza de los datos recopilados en cuanto a su formato y se comenzó a utilizar herencia mediante la creación de una clase abstracta con los métodos comunes de las clases concretas. Este aspecto mejora enormemente la extensibilidad y facilidad de mantenimiento del código para situaciones futuras.

Gran parte del tiempo del sprint, se dedicó a la implantación de \textit{testing}. Esto se debe a que los test unitarios en los proyectos de desarrollo de software tienen un valor crucial en la detección de errores y mejora de mantenimiento. Además, aseguran que el software tenga una cierta calidad. Para este proceso se utilizaron herramientas como MagicMock y UnitTest ~\cite{unittest:latex}.

Por otro lado, se creó un entorno virtual en Python con \textit{virtualenv} para la gestión de dependencias y de esta manera aislar el entorno en el que se está desarrollando el proyecto ~\cite{virtualenv:latex}.

Finalmente, se creó un directorio en el proyecto con un fichero en el que se realizaban las llamadas a las correspondientes clases en las que se desarrolló el \textit{web scraping} de cada una de las universidades para recopilar la información. Todos estos datos fueron recopilados y se realizó una conexión con la base de datos para que fuesen almacenados ahí de forma segura.

\imagensprint{sprint4}{Sprint 4.}

\subsection{Sprint 5 (04/04/2024 -
17/04/2024)}
Los primeros días de este periodo, consistieron en corregir algunos errores que se percibieron al ejecutar los test de aspectos que no se estaban realizando del todo bien en el proceso de \textit{web scraping}. Una vez ya realizada esta tarea, se realizaron las inserciones de los datos recopilados con un formato adecuado a una base de datos MySQL ~\cite{mysqlpython:latex}. Esta base de datos puede ser considerado un almacén de datos o \textit{Data Warehouse} en el que aparecían los atributos comunes y no comunes entre las distintas webs de las distintas universidades junto con una recopilación de todas las convocatorias.

Una vez realizado esto, se comenzó con el desarrollo de la web que iba a recoger todas estas convocatorias del PDI y PAS. Para realizar esta web se decidió optar por un \textit{framework} como Asp .NET Core MVC el cual es utilizado para realizar aplicaciones web siguiendo el patrón Modelo-Vista-Controlador ~\cite{aspnetcore:latex}.

Por otro lado, también se realizaron algunas tareas de \textit{mockup} pensando en diseños atractivos para la aplicación web a realizar. En el momento que se encontraron una serie de prototipos acordes con la idea principal, se comenzó a trabajar con HTML y CSS para la realización de las interfaces.

Finalmente, se utilizaron \textit{media queries} para que la web tuviera un diseño \textit{responsive}  y, de esta manera, que se pudiese utilizar en cualquier tipo de pantalla sin importar las dimensiones de esta.

\imagensprint{sprint5}{Sprint 5.}

\subsection{Sprint 6 (18/04/2024 -
01/05/2024)}

Durante este \textit{sprint}, se realizaron con diseño adaptativo las ventanas en las que aparecen los detalles de cada una de las convocatorias.

Posteriormente, para añadir funcionalidad extra a mi aplicación web, se decidió integrar el inicio de sesión con Google mediante autenticación basada en OAuth 2.0 ~\cite{autenticaciongoogle:latex}. Esto se realizó para que se pudiese permitir la integración con Google Calendar desde la web y de esta manera aportar un valor extra a la aplicación.

La integración con Google Calendar también se realizó en este sprint ~\cite{googlecalendar:latex}. De esta manera, los usuarios que iniciasen sesión con la aplicación podrán añadir a su calendario la fecha en la que caduca esa convocatoria y así poder acordarse de realizar solicitudes en el caso de que estén interesados.

Con la finalización de este \textit{sprint} se consideraba que la funcionalidad de la aplicación se había acabado de desarrollar en su totalidad.

\imagensprint{sprint6}{Sprint 6.}

\subsection{Sprint 7 (02/05/2024 -
15/05/2024)}
Esta iteración, fue una de las finales del proyecto y como consecuencia, se dedicó gran parte del tiempo para realizar la documentación. Para ello, se recopilaron todas las anotaciones, datos y tareas de los anteriores \textit{sprints}. 

Por otro lado, se ha comenzado a dedicar gran parte del tiempo a la investigación sobre las distintas posibilidades que hay para lanzar la aplicación de manera gratuita para que pueda ser probada por cualquier usuario.

Por último, dado que la integración con Google Calendar solo era viable realizarla con usuarios de \textit{testing}, se decidió eliminar esta integración e incorporar a la web un sistema de avisos por correo en la que se enviaban adjuntos archivos ICalendar para que cada usuario pudiese añadir sus convocatorias de interés como eventos a sus calendarios personales ~\cite{icalendar:latex}.

En este sprint, al igual que en la mayoría de los anteriores, también se dedicó parte de tiempo a corrección de errores y mejoras.
\imagensprint{sprint7}{Sprint 7.}


\subsection{Sprint 8 (16/05/2024 -
30/05/2024)}
En esta etapa, se continuó trabajando con la documentación del proyecto. Además, pese a tener algunos problemas con algunas herramientas utilizadas para hacer el \textit{deployment}, se acabó realizando satisfactoriamente mediante la utilización de un plan gratuito de Azure Services ~\cite{azureservices:latex}. 

Por otro lado, se realizó la implementación de un \textit{cron} para mantener la base de datos actualizada. Esto lo pude hacer gracias a GitHub Actions ~\cite{githubactions:latex}.

Finalmente, también se realizó alguna mejora en el código. Hasta ahora, para la importación de paquetes se estaba trabajando con rutas relativas y para un mejor despliegue, se comenzaron a utilizar rutas absolutas. Este es uno de los cambios más destacables en cuanto a código en esta iteración.
\imagensprint{sprint8}{Sprint 8.}

\subsection{Sprint 9 (31/05/2024 -
12/06/2024)}
Con esta iteración se pondría broche final al proyecto. Ya con el despliegue de la aplicación realizado, se dedicó la mayoría del tiempo a la preparación de la documentación y últimos detalles del proyecto para entregarlo al tribunal.

Se realizaron las últimas mejoras en el código,  y se realizó una carga del proyecto en una máquina virtual. Esta máquina virtual se almacenará en un USB para su posterior entrega al tribunal.

\imagensprint{sprint9}{Sprint 9.}

\subsection{Resumen}

Una vez finalizados todos los \textit{sprints} se han podido obtener una serie de conclusiones sobre cómo ha sido la planificación del proyecto y el tiempo que se ha invertido en cada una de las distintas actividades y tareas, estas conclusiones se van a comentar a continuación:

\begin{itemize}
\item En las primeras iteraciones, como es lógico, se invirtió gran parte del tiempo a la planificación y estructuración del proyecto.

\item En casi todos los \textit{sprints} se ha dedicado un porcentaje de tiempo a la programación y a la realización de nuevos desarrollos.

\item El soporte y las mejoras de la aplicación, fue algo frecuente en gran parte de las iteraciones gracias a la implementación de \textit{testing} y realización de test manuales mediante el uso de la aplicación, se pudieron corregir los errores.

\item Una vez que acabaron los desarrollos en cuanto a la extracción de datos, se dedicó más tiempo al prototipado y al desarrollo de las interfaces de la aplicación.

\item Los últimos tramos del proyecto fueron dedicados a la documentación de la aplicación y a su lanzamiento.

\end{itemize}

\section{Estudio de viabilidad}

\subsection{Viabilidad económica}

\subsubsection{Introducción}
En este apartado se va a realizar un análisis de los costes y los beneficios que tendría este proyecto. Estos factores serán muy importantes en el caso de que este proyecto se ejecutase en un entorno real. De esta manera podríamos examinar su viabilidad financiera e incluso apreciar cual sería el impacto que tienen los costes y beneficios a la hora de tomar decisiones.

Por otro lado, dado que el proyecto no se ha llevado a cabo  en un entorno real, podría haber otros factores que afecten a la viabilidad económica del mismo y que no hayan sido detallados en este apartado.

\subsubsection{Análisis de los Costes}

En este apartado, podemos identificar tres categorías de costes que casi todo proyecto software se encuentra en el camino:  los costes humanos, los costes de hardware y los costes de software.

\textbf{Costes humanos:}

Este proyecto ha sido realizado por una sola persona. Considerando que ese empleado es un Ingeniero Informático cuyo país de residencia es España, según el portal \textit{Talent.com} al que puedes acceder desde el siguiente \href{https://es.talent.com/salary?job=ingeniero+inform%C3%A1tico}{enlace}, el salario promedio por hora es de 13,86 € ~\cite{salarioingeniero:latex}. Contando que se han dedicado 300 horas, podemos deducir lo siguiente:

\[ 13,86 \, \text{€} \times 300 \, \text{horas} = 4158 \, \text{€ brutos totales} \]

Esta cantidad la tenemos que dividir entre los 4 meses en los que se ha realizado en proyecto y por lo tanto:

\[ 4158 \, \text{€} / 4 \, \text{meses} = 1039,5 \, \text{€ brutos mensuales} \]

Una vez analizado lo que va a recibir el trabajador, vamos a analizar cuáles serían, como consecuencia, los costes totales de personal que tendrá la empresa ~\cite{costetrabajador:latex}:
 

\begin{table}[H]
    \centering
    \renewcommand{\arraystretch}{1.2}
    \setlength{\tabcolsep}{20pt}
    \begin{tabular}{l l}
        \hline
        \textbf{Concepto} & \textbf{Coste} \\ \hline
        Salario mensual bruto & 1039,5€ \\
        Contingencias comunes (23,6\%) & 245,32€\\
        Tipo general de  desempleo (5,5\%) & 57,17€\\
        Contingencias profesionales (3,5\%) & 36,38€\\
        Fondo de garantía social (0,2\%)& 2,08€\\
        Formación profesional (0,6\%) & 6,24€\\
        Salario mensual total & 1386,69€\\  \hline
        \textbf{Total costes humanos} & \textbf{5546,76€}\\ \hline
    \end{tabular}
    \caption{Costes humanos totales}
    \label{tab:costes_humanos}
\end{table}

\newpage

\textbf{Costes hardware:}

Para el desarrollo del proyecto, tan solo es necesario poseer de un ordenador portátil. En este caso, suponemos que el ordenador ha sido comprado nuevo por la empresa para la realización de este nuevo proyecto. Contando con que el proceso de \textit{web scraping} es costoso en cuanto a rendimiento, se necesitará un ordenador con unas características medianamente potentes, por lo tanto, el coste hardware será el siguiente:

\begin{table}[H]
    \centering
    \renewcommand{\arraystretch}{1.2}
    \setlength{\tabcolsep}{20pt}
    \begin{tabular}{l l}
        \hline
        \textbf{Concepto} & \textbf{Coste} \\ \hline
        Ordenador Portátil & 879,99€ \\  \hline
        \textbf{Total costes hardware} & \textbf{879,99€}\\ \hline
    \end{tabular}
    \caption{Costes hardware}
    \label{tab:costes_hardware}
\end{table}

\textbf{Costes software:}

En cuanto a los costes software, tan solo podríamos recalcar el sistema operativo utilizado, pero en este caso, la licencia venía con el ordenador portátil de fábrica. Este sistema operativo era el Windows 10 Pro.

Por otro lado, me gustaría destacar que todas las herramientas software utilizadas son gratuitas. Esto es un factor muy a tener en cuenta a la hora de realizar un estudio económico del proyecto dado que reduce enormemente los costes en comparación con otros proyectos.

\textbf{Otros costes:}
Aunque este tipo de costes no estén encasillados en una categoría como las definidas previamente. Un proyecto de este estilo suele necesitar de otros productos o servicios que también son imprescindibles para que el proyecto tenga éxito.  

Los productos y servicios mencionados previamente se consideran los siguientes:

\begin{itemize}
\item \textbf{Infraestructura de alojamiento y dominio: } Dado que para este tipo de proyectos es necesario disponer de instancias de computación en la nube y un dominio personalizado.
\item \textbf{Conexión a internet: } Debido a que a pesar de que el software sea gratuito, se necesita en ocasiones conexión con la red para poder ejecutarlo.
\item \textbf{Documentación impresa y material para la presentación: } Dado que es necesario por si se quieren tener reuniones con posibles compradores del producto o para presentarlo dentro de la empresa en la que se ha realizado el proyecto. En este caso esto se ha realizado para poder presentarlo ante el tribunal.

\end{itemize}

\begin{table}[H]
    \centering
    \setlength{\tabcolsep}{20pt}
    \begin{tabular}{l l}
        \hline
        \textbf{Concepto} & \textbf{Coste} \\ \hline
        Infraestructura y Dominio & 65,55€ \\
        Internet  & 87,60€\\
        Documentación y material para la presentación & 50€\\  \hline
        \textbf{Total otros costes} & \textbf{203,15€}\\ \hline
    \end{tabular}
    \caption{Otros Costes}
    \label{tab:otros_costes}
\end{table}


\textbf{Costes totales:}

Una vez analizados todos los costes anteriores, podemos calcular cuales han sido los costes totales:

\begin{table}[H]
    \centering
    \renewcommand{\arraystretch}{1.2}
    \setlength{\tabcolsep}{20pt}
    \begin{tabular}{l l}
        \hline
        \textbf{Concepto} & \textbf{Coste} \\ \hline
        Costes Humanos & 5546,76€ \\
        Costes Software & 0€ \\
        Costes Hardware  & 879,99€\\
        Otros costes & 203,15€\\  \hline
        \textbf{Costes Totales} & \textbf{6629,9€}\\ \hline
    \end{tabular}
    \caption{Costes Totales}
    \label{tab:costes_totales}
\end{table}

\textbf{Beneficios:}

El objetivo de esta aplicación ha sido aportar un servicio a las universidades de Castilla y León mediante una aplicación en la que se pudiesen ver recopiladas todas las convocatorias que estas universidades publican en sus webs. 

Por esta razón, en caso de que la aplicación pasase a desarrollo, esta misma se distribuirá de manera gratuita para un beneficio de la comunidad universitaria.

Por otro lado, si se quisiese obtener un beneficio económico de la misma, se podría plantear la implantación de anuncios y publicidad. Otra opción válida como modelo de negocio, sería el cobro de un importe para aquellos usuarios que hayan accedido a una convocatoria gracias a esta web y hayan sido elegidos para el puesto solicitado.

\subsection{Viabilidad legal}
Al igual que la viabilidad económica, la viabilidad legal también es un aspecto muy importante para considerar a la hora de planificar un proyecto software. Este apartado va a consistir en el análisis del cumplimiento de leyes por parte del proyecto y las licencias de software. 

En primer lugar, en cuanto al contenido multimedia utilizado, ha sido obtenido en distintas webs cuyos contenidos están libres de \textit{copyright} bajo la licencia \textbf{CC0 (Creative Commons Zero)} ~\cite{codigoabierto:latex}. Esto implica que ese contenido multimedia puede ser utilizado sin tener permisos debido a que no están protegidas por derechos de autor. En este caso, se ha dado alguna excepción a la hora de obtener los logos de las distintas universidades involucradas en el proyecto.

Estas imágenes han sido obtenidas a través de las webs de las propias universidades. En principio, dado que esto es un proyecto educativo, estas imágenes pueden ser utilizadas sin ningún inconveniente. Además, en el caso de que el proyecto pasase a desarrollo, no se prevé tener ningún inconveniente en este aspecto legal con las mismas dado que los beneficiarios de que esta aplicación pasase a desarrollo serían las propias universidades.

Posteriormente, se van a analizar las licencias de las dependencias utilizadas en el proyecto. Este proceso lo vamos a dividir en 2 partes. En primer lugar se van a analizar las dependencias utilizadas en cuanto al proceso de \textit{web scraping} realizado con Python. Estas dependencias se han aparecerán listadas en el fichero \textit{requirements.txt} del proyecto y serán las siguientes:

\begin{table}[H]
    \centering
    \renewcommand{\arraystretch}{1.2} % Ajusta el espacio entre filas
    \begin{tabularx}{\textwidth}{l l X l}
        \hline
        \textbf{Dependencia} & \textbf{Versión} & \textbf{Descripción} & \textbf{Licencia} \\
        \hline
        autopep8 & 2.0.4 & Guía de estilos para Python & MIT \\
        beautifulsoup4 & 4.12.3 & Biblioteca para la extracción de datos en HTML & MIT \\
        mysql-connector-python & 8.3.0 & Herramienta para conexión con la base de datos & GPL \\
        pillow & 10.3.0 & Biblioteca para el tratamiento de imágenes & BSD \\
        pytest & 8.1.1 & Herramienta para implementación de testing & MIT \\
        requests & 2.31.0 & Biblioteca para hacer peticiones HTTP & Apache2 \\
        selenium & 4.18.1 & Herramienta automatización sitio web & Apache2 \\
        \hline
    \end{tabularx}
    \caption{Dependencias junto con sus Licencias Python}
    \label{tab:licencias1}
\end{table}

\newpage
Posteriormente, veremos los paquetes que han sido instalados en .NET para la realización de la web junto con sus respectivas licencias:

\begin{table}[H]
    \centering
    \renewcommand{\arraystretch}{1.2} % Ajusta el espacio entre filas
    \begin{tabularx}{\textwidth}{X l X l}
        \hline
        \textbf{Dependencia} & \textbf{Versión} & \textbf{Descripción} & \textbf{Licencia} \\
        \hline
        X PagedList Mvc Core & 9.1.2 & Herramienta para aplicar paginación & MIT \\
        Pomelo Entity Framework Core MySql& 8.0.2 & Herramienta para la conexión con la base de datos & MIT \\
        MySql Entity Framework Core & 8.0.2 & Herramienta para conexión con la base de datos & MIT \\
        Microsoft
        Entity Framework Core & 8.0.3 & Herramienta para acceso a bases de datos & Apache2 \\
        Microsoft Entity Framework Core Tools & 8.0.3 & Herramienta para el trabajo con bases de datos mediante objetos & Apache2 \\
        Mailkit & 4.6.0 & Herramienta envío de emails & Apache2 \\
        \hline
    \end{tabularx}
    \caption{Dependencias junto con sus Licencias .NET}
    \label{tab:licencias2}
\end{table}

Tal y como hemos podido apreciar anteriormente, la mayoría de las dependencias de este proyecto tienen una licencia de software MIT o Apache v2.0 ambas licencias son bastante permisivas y disponen de gran disponibilidad en cuanto a su uso.

Una vez evaluado lo anterior, se ha decidido que el código fuente sea público para que pueda ser modificado, estudiado y utilizado por cualquier persona sin restricciones de propiedad intelectual o de licencia. Por lo tanto, la licencia otorgada al proyecto será una licencia \textbf{MIT} esta licencia permite a los usuarios utilizar, modificar y distribuir el software sin restricciones, siempre y cuando se proporcione atribución al autor original y se incluya un aviso de derechos de autor en todas las copias del software ~\cite{licenciamit:latex}.

Además, otro aspecto importante a considerar es que esta licencia nos permite compatibilidad con Apache v2.0. Algo que puede ser ventajoso de cara al futuro y nuevas implementaciones.

Mediante la elección de esta licencia se contribuye enormemente con la comunidad dado que el producto se podrá utilizar y compartir con total libertad.


