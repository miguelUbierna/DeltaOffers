\apendice{Especificación de diseño}

\section{Introducción}

En esta apartado, se va a constatar las especificaciones de diseño de este proyecto. Estas especificaciones de diseño van acorde a los objetivos propuestos al principio del proyecto y a los requisitos funcionales y no funcionales mencionados previamente. El diseño del sistema, lo podemos dividir en tres partes:

\begin{itemize}
    \item \textbf{Diseño de datos:} Define como se estructuran y almacenan los datos del sistema. Este proceso es muy importante para entender cómo se relacionan los datos y cómo definir una estructura ventajosa para el programa y sus correspondientes procedimientos.
    \item \textbf{Diseño de procedimental:} Indica los algortitmos y procedimientos utilizados en el sistema. En este proceso, suelen utilizarse los diagramas de flujo como representación gráfica de los algoritmos utilizados. 
    \item \textbf{Diseño de arquitectónico:} Que el sistema posea de una buena arquitectura, es ernormemente venatjoso para que sea robusto y eficiente. En este apartado se especificará cual es la arquitecura del sistema y sus principales componentes. Además, se explicarán los patrones de diseño aplicados en el proyecto.
\end{itemize}

\section{Diseño de datos}

La aplicación solo cuenta con una entidad 'Convocatorias' , se tomo la decisión de solo tener una entidad dado que solo con una se pueden cumplir los objetivos de la aplicación. 

Los atributos e información de interés de las correspondientes convocatorias se pueden encapsular en una sola tabla sin necesidad de relaciones con otras entidades lo cual podría suponer una pérdida de eficiencia cuando se acceden a los datos.

A pesar de que para completar la funcionalidad de nuestra aplicación solo se necesita una entidad, a continuación se va a mostrar cual es y los atributos que tiene.

\imagenentidad{EntidadTFG}{Entidad Convocatorias.}

\section{Diseño procedimental}

En este apartado se van a detallar los procedimientos que va a realizar un usuario al utilizar la aplicación. Para ello, se va a realizar un diagrama de flujo para indicar de manera esquematizada los distintos procedimientos y flujos que el usuario puede seguir al utilizar la aplicación 

\newpage
\begin{landscape}
\subsection{Diagrama de flujo}
\imagenapaisada{DiagramaDeFlujo}{Diagrama de flujo.}
\end{landscape}
\newpage

\section{Diseño arquitectónico}
La arquitectura software es la encargada de diseñar la estructura que va a tener el sistema. Tener una arquitectura consistente es fundamental en aspectos de eficiencia, mantenibilidad y seguridad.

En esta fase de estructuración del sistema, juegan un papel muy importante los patrones de diseño los cuales proporcionan una mayor eficiencia y calidad de software.
Además, la utilización de estos patrones mejora la escalabilidad de los proyectos permitiendo que estos puedan crecer pero continuando con una estructura sólida.

En este apartado, se van a comentar los patrones de diseño que han sido utilizados a lo largo del proyecto. Algunos de estos patrones han sido aplicados a la parte de web scraping mientras que otros han sido aplicados a la hora de realizar la web. Todos los patrones utilizados, se comentarán a continuación.


\subsection{Patrones de diseño y arquitectura}

\subsubsection{Patrón de Arquitectura MVC (Modelo-Vista-Controlador)}
Este patrón se utiliza que la interfaz de usuario, los datos y la logica de la aplicación, trabajen de manera desacoplada. Con este patrón utilizado para aplicaciones web, las solicitudes se enrutan a un controlador que es el que trabaja con el modelo de datos y finalmente elige la vista que se va a mostrar pasadole el modelo de datos. Más detalladamente, las capas de este paton consisten en lo siguiente:

\begin{itemize}
    \item \textbf{Modelo:} en esta capa se encontrará una representación de los datos del dominio y la lógica de negocio de la aplicación con las que podemos gestionar entidades. El modelo es el que nos proporciona los datos que utilizamos en la aplicación y nos permite almacenarlos correctamente.
    \item \textbf{Vista:} La vista es la que se encarga de que se visualicen las interfaces de nuestra aplicación. En la vista se podrán visualizar los datos del modelo que la vista ha recibido a través del controlador.
    Para esta parte de la vista, se han utilizado las \textit{Razor Pages} dado que nos permiten incluir logica en C\# dentro del HTML.
    \item \textbf{Controlador:} Este actua como intermediario entre el usuario y el sistema. Este es el encargado de enviar información desde el modelo a la vista y viceversa.
\end{itemize}

\begin{figure}[H]
    \centering
    \includegraphics[width=0.9\linewidth]{DocumentacionTFG//img/PatronMVC.png}
    \caption{Patrón Modelo-Vista-Controlador}
\end{figure}

\subsubsection{Patrón Inyección de dependencias}

Este es un patrón de diseño admitido por .NET en el que las dependencias de una determinada clase, no necesitan ser creadas si no que se inyectan directamente. Esto hace que las clases de alto nivel, no dependan de las clases de bajo nivel.

Esto reduce el acomplamiento entre los distintos componentes de la aplicación y mejora la mantenibilidad del código.

Un ejemplo básico de utilización de este patrón en nuestra aplicación aparece cuando inyectamos el contexto de datos dentro de nuestros controladores. En lugar de generar una instancia en el propio controlador, se recibe una instancia de una determinada clase proporcionada por el contenedor de servicios donde se configuran todos los servicios que se van a utilizar en la aplicación.

\begin{figure}[H]
    \centering
    \includegraphics[width=0.7\linewidth]{DocumentacionTFG//img/InyeccionDependencias.PNG}
    \caption{Inyección de dependencias}
\end{figure}

\subsubsection{Patrón MVVM (Modelo-Modelo de Vista-Modelo)}

Este patrón aisla la vista del modelo y el modelo de la vista. La función principal del modelo de vista en este caso es proporcionar una represantación de los datos que se adapte a la vista. Esto permite separar la lógica de presentación de la lógica de negocio y por lo tanto se evita que se realicen cambios importantes en el código del modelo.

En .NET este patrón se ha aplicado creando una serie de clases \textit{ViewModel} en las cuales solo se incluian los atributos que fuesen necesarios para una vista en concreto. De esta manera se evitan realizar cambios en el modelo de datos.

\begin{figure}[H]
    \centering
    \includegraphics[width=0.9\linewidth]{DocumentacionTFG//img/PatronMVVM.png}
    \caption{Patrón MVVM}
\end{figure}

\subsubsection{Patrón Estrategia}
Este patrón se aplicó en el proceso de web scraping, su principal proposito es permitir al objeto cliente elegir cual de las estrategias le conviene.

En el caso del proyecto, se dispone una serie de algoritmos de scraping en función de cada universidad. Estos algoritmos estarán cada uno en clases separadas pero todos ellos implementarán una interfaz común. Esto hace que según la universidad de la que se quiera hacer el scriping, se selecciona su estrategia.

El patrón estrategia puede ser muy interesante de cara a futuro si se incluyen nuevas universidades en la aplicación dado que facilita enormemente las nuevas implementaciones y el testeo.
\begin{figure}[H]
    \centering
    \includegraphics[width=0.7\linewidth]{DocumentacionTFG/img/PatronEstrategia.png}
    \caption{Patrón Estrategia}
\end{figure}

\section{Diseño de interfaces}
Dado que no habia utilizado ninguna herramienta de \textit{mockup} recientemente y me parecía demasiado arcaico realizar diseños de interfaces a mano, decidí realizar los prototipos en sucio con PowerPoint. Esta decisión se tomó debido a la facilidad de uso que proporciona esa herramienta. 

Estos prototipos se fueron creando mediante la utilización de formas, contornos, fuentes, etc. A pesar de que la herramienta utilizada no es especifica para realizar los protoripos, el resultado fue satisfactorio.

\begin{figure}[H]
    \centering
    \includegraphics[width=0.9\linewidth]{DocumentacionTFG//img/MaestroDiseño.png}
    \caption{Diseño Ventana Maestra}
\end{figure}

\begin{figure}[H]
    \centering
    \includegraphics[width=0.9\linewidth]{DocumentacionTFG//img/DetalleDiseño.png}
    \caption{Diseño Ventana Detalle}
\end{figure}

\begin{figure}[H]
    \centering
    \includegraphics[width=0.9\linewidth]{DocumentacionTFG//img/DiseñoAdaptativo.png}
    \caption{Diseño Adaptativo}
\end{figure}

Estos prototipos surgieron como idea inicial. Las interfaces reales se han ido modificando a medida que se iban programando para que seán mas atractivas para el usuario.
